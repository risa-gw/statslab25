% Options for packages loaded elsewhere
\PassOptionsToPackage{unicode}{hyperref}
\PassOptionsToPackage{hyphens}{url}
\PassOptionsToPackage{dvipsnames,svgnames,x11names}{xcolor}
\documentclass[
  12,
]{article}
\usepackage{xcolor}
\usepackage[margin=1in]{geometry}
\usepackage{amsmath,amssymb}
\setcounter{secnumdepth}{-\maxdimen} % remove section numbering
\usepackage{iftex}
\ifPDFTeX
  \usepackage[T1]{fontenc}
  \usepackage[utf8]{inputenc}
  \usepackage{textcomp} % provide euro and other symbols
\else % if luatex or xetex
  \usepackage{unicode-math} % this also loads fontspec
  \defaultfontfeatures{Scale=MatchLowercase}
  \defaultfontfeatures[\rmfamily]{Ligatures=TeX,Scale=1}
\fi
\usepackage{lmodern}
\ifPDFTeX\else
  % xetex/luatex font selection
\fi
% Use upquote if available, for straight quotes in verbatim environments
\IfFileExists{upquote.sty}{\usepackage{upquote}}{}
\IfFileExists{microtype.sty}{% use microtype if available
  \usepackage[]{microtype}
  \UseMicrotypeSet[protrusion]{basicmath} % disable protrusion for tt fonts
}{}
\usepackage{graphicx}
\makeatletter
\newsavebox\pandoc@box
\newcommand*\pandocbounded[1]{% scales image to fit in text height/width
  \sbox\pandoc@box{#1}%
  \Gscale@div\@tempa{\textheight}{\dimexpr\ht\pandoc@box+\dp\pandoc@box\relax}%
  \Gscale@div\@tempb{\linewidth}{\wd\pandoc@box}%
  \ifdim\@tempb\p@<\@tempa\p@\let\@tempa\@tempb\fi% select the smaller of both
  \ifdim\@tempa\p@<\p@\scalebox{\@tempa}{\usebox\pandoc@box}%
  \else\usebox{\pandoc@box}%
  \fi%
}
% Set default figure placement to htbp
\def\fps@figure{htbp}
\makeatother
% definitions for citeproc citations
\NewDocumentCommand\citeproctext{}{}
\NewDocumentCommand\citeproc{mm}{%
  \begingroup\def\citeproctext{#2}\cite{#1}\endgroup}
\makeatletter
 % allow citations to break across lines
 \let\@cite@ofmt\@firstofone
 % avoid brackets around text for \cite:
 \def\@biblabel#1{}
 \def\@cite#1#2{{#1\if@tempswa , #2\fi}}
\makeatother
\newlength{\cslhangindent}
\setlength{\cslhangindent}{1.5em}
\newlength{\csllabelwidth}
\setlength{\csllabelwidth}{3em}
\newenvironment{CSLReferences}[2] % #1 hanging-indent, #2 entry-spacing
 {\begin{list}{}{%
  \setlength{\itemindent}{0pt}
  \setlength{\leftmargin}{0pt}
  \setlength{\parsep}{0pt}
  % turn on hanging indent if param 1 is 1
  \ifodd #1
   \setlength{\leftmargin}{\cslhangindent}
   \setlength{\itemindent}{-1\cslhangindent}
  \fi
  % set entry spacing
  \setlength{\itemsep}{#2\baselineskip}}}
 {\end{list}}
\usepackage{calc}
\newcommand{\CSLBlock}[1]{\hfill\break\parbox[t]{\linewidth}{\strut\ignorespaces#1\strut}}
\newcommand{\CSLLeftMargin}[1]{\parbox[t]{\csllabelwidth}{\strut#1\strut}}
\newcommand{\CSLRightInline}[1]{\parbox[t]{\linewidth - \csllabelwidth}{\strut#1\strut}}
\newcommand{\CSLIndent}[1]{\hspace{\cslhangindent}#1}
\setlength{\emergencystretch}{3em} % prevent overfull lines
\providecommand{\tightlist}{%
  \setlength{\itemsep}{0pt}\setlength{\parskip}{0pt}}
\usepackage{setspace}\doublespacing
\usepackage{indentfirst}
\usepackage{bookmark}
\IfFileExists{xurl.sty}{\usepackage{xurl}}{} % add URL line breaks if available
\urlstyle{same}
\hypersetup{
  pdftitle={Your Latex Template for Replication},
  pdfauthor={Your Name},
  colorlinks=true,
  linkcolor={Maroon},
  filecolor={Maroon},
  citecolor={Blue},
  urlcolor={blue},
  pdfcreator={LaTeX via pandoc}}

\title{Your Latex Template for Replication}
\usepackage{etoolbox}
\makeatletter
\providecommand{\subtitle}[1]{% add subtitle to \maketitle
  \apptocmd{\@title}{\par {\large #1 \par}}{}{}
}
\makeatother
\subtitle{Subtitle}
\author{Your Name}
\date{Date}

\begin{document}
\maketitle

\pagenumbering{gobble} 
\centerline{}
\pagenumbering{arabic}

\section{Introduction}\label{introduction}

This document provides the basic RMarkdown template using LaTeX for the
final replication project and (hopefully) your future work. There are
also lots of RMarkdown resources online that you should feel free to
reference, such as
\href{https://rstudio.github.io/cheatsheets/html/rmarkdown.html}{this
page} and \href{https://rmarkdown.rstudio.com/lesson-15.HTML}{this
cheatsheet}.

\subsection{Second Layer}\label{second-layer}

This is the second layer of Introduction. You can keep creating deeper
layers with more hashtags in front of the title.

\section{Mathematical Equations}\label{mathematical-equations}

\begin{itemize}
\item
  There are two modes for mathematical expressions: the \textbf{inline
  mode} and the \emph{display mode}. The first one is used to write
  formulas that are part of a text. The second one is used to write
  expressions that are on separate lines.
\item
  For example, to type an inline equation for calculating sample mean:
  \(\hat{\mu} = \bar{y} = \frac{\sum_{i = 1}^{n}y_i}{n}\)
\item
  Or type equation 1 in display mode using:
\end{itemize}

\[
\bar{y} = \frac{y_1 + y_2 + y_3 + ... + y_n}{n} = \frac{\sum_{i = 1}^{n} y_i}{n}
(\#eq:mean)
\]

\begin{itemize}
\item
  You can then refer to the equation in text using
  \texttt{\textbackslash{}@ref(eq:mean)}
\item
  You can always search symbols you don't know the LaTeX code for or use
  an online LaTeX generator such as this
  \href{https://latexeditor.lagrida.com/}{one}
\end{itemize}

\section{Figures}\label{figures}

You can customize how your figures appear with the header of the code
chunk, or with global options in the \texttt{knitr::opts\_chunk\$set()}
commands above.

\begin{figure}

{\centering \includegraphics{final_project_template_files/figure-latex/unnamed-chunk-1-1} 

}

\caption{Example plot title}\label{fig:unnamed-chunk-1}
\end{figure}

\section{Citations in RMarkdown}\label{citations-in-rmarkdown}

\begin{itemize}
\item
  There is a .bib file in the GitHub ``Final Project'' folder that has
  the paper. You should input your references there. If you use a
  reference software, such as Zotero or Mendeley, there are options to
  export a bib file from it for your project or specific files as well.
\item
  To cite an article, write \texttt{@article\_key} such as Mandel and
  Semyonov (\citeproc{ref-mandel_going_2016}{2016}) or for parenthesis
  (\citeproc{ref-mandel_going_2016}{Mandel and Semyonov 2016:1045})
\item
  To edit the citation style, change the citation language style (CSL)
  specified in the YAML header.
\item
  RMarkdown will automatically add a references section at the end of
  your document.
\item
  For more information on RMarkdown references, see guides
  \href{https://bookdown.org/yihui/bookdown/citations.html}{here} and
  \href{https://bookdown.org/yihui/rmarkdown-cookbook/bibliography.html}{here}
\end{itemize}

\section*{References}\label{bibliography}
\addcontentsline{toc}{section}{References}

\phantomsection\label{refs}
\begin{CSLReferences}{1}{1}
\bibitem[\citeproctext]{ref-mandel_going_2016}
Mandel, Hadas, and Moshe Semyonov. 2016. {``Going {Back} in {Time}?
{Gender} {Differences} in {Trends} and {Sources} of the {Racial} {Pay}
{Gap}, 1970 to 2010.''} \emph{American Sociological Review}
81(5):1039--68. doi:
\href{https://doi.org/10.1177/0003122416662958}{10.1177/0003122416662958}.

\end{CSLReferences}

\end{document}
